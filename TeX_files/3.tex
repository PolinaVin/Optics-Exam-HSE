\section{Преломление на сферической поверхности. Фокусы сферичекой поверхности. Изображение предмета. Преломление в линзе. Общая формула линзы (вывод). Фокусные расстояния тонкой линзы. Действительное и мнимое изображения. Линейное (поперечное) увеличение. Оптическая сила линз.}

Сначала введём необходимые определения:
\begin{enumerate}

\item{\textbf{Оптическая система} — совокупность оптических элементов (преломляющих, отражающих), созданная для преобразования световых пучков.}

\item{\textbf{Оптическая ось системы} — прямая линия, являющаяся осью симметрии преломляющих (отражающих) поверхностей. Она проходит перпендикулярно этим поверхностям через их центр кривизны.}

\item{\textbf{Центрированная оптическая система} — совокупность однородных преломляющих и отражающих сред, отделённых друг от друга симметричными поверхностями, центры кривизны которых находятся на одной прямой. Эту прямую называют \textit{главной оптической осью системы}.}

\item{\textbf{Гомоцентрический пучок} — пучок, все образующие лучи которого при своём продолжении сходятся в одной точке. \textit{Параллельный} пучок тоже считается гомоцентрическим, т.к. он исходит из бесконечно удалённой точки.}

\item{\textbf{Параксиальный пучок} — пучок, все образующие лучи которого распространяются вдоль оси центрированной оптической системы и образуют малые углы с осью и нормалями к преломляющим и отражающим поверхностям.}

\item{Пусть оптическая система преобразует свет, не нарушая гомоцентричности пучка, причём пучок лучей, исходящий из точки $P$, сходится в точке $P'$. Тогда точка $P'$ — это \textbf{изображение} точки $P$.}


\item{Изображение называется \textbf{действительным}, если световые лучи от точки $P$ сходятся к точке $P'$ при своём распространении. Если же в точке $P'$ сходятся продолжения лучей в направлении, обратном направлению распространения света, то изображение называется \textbf{мнимым}.}

\item{Пусть  на  оптическую  систему  падает  пучок  лучей, параллельных  главной  оптической  оси.  \textbf{Задний (второй главный) фокус} -- это точка пересечения пучка таких лучей (или их  продолжений)  на  выходе  из  системы.}

\item{Если на систему падает  пучок лучей, исходящий из некоторой  точки,  и  после  прохождения  оптической  системы  лучи  идут параллельно главной оптической оси, исходная точка называется \textbf{передним (первым главным) фокусом}. Передний и задний фокусы всегда лежат на главной оптической оси.}

\item{\textbf{Линейное (поперечное) увеличение} — отношение линейных размеров изображения и предмета:
\begin{equation}
\beta = \frac{y'}{y}.
\label{eq:lin_magn}
\end{equation} 

Отрезки $y$ и $y'$ считаеются положительными, если они откладываются вверх от оптической оси, и отрицательными — в противном случае.}

\item{Если увеличение положительное, то изображение \textbf{прямое}. В противном случае изображение \textbf{обратное}.}

\item{Две сопряжённые плоскости, отображающиеся с линейным увеличением $\beta = 1$, называются \textit{главными}. Точки пересечения главных плоскостей с главной оптической осью — \textit{главные точки} оптической системы.}

\item{\textbf{Главные фокусные расстояния} — расстояния от главных точек до соответствующих фокусов. Эти расстояния считаются положительными, если свет идёт от главной плоскости к соответствующему главному фокусу (правило знаков).}


\end{enumerate}

\subsection{Преломление на сферической поверхности}
 
Пусть две среды с разными показателями преломления ($n$ и $n'$) отделены друг от друга сферической границей $S$ радиуса $R$. Эта граница может рассматриваться как оптическая система, преобразующая падающее на неё излучение предмета. Считаем радиус сферы положительным, если её центр находится с той стороны, куда распространяются лучи.

\begin{figure}[H]
	\centering
	\includegraphics*[width=0.7\textwidth]{3_1}
\end{figure}


Рассмотрим луч $OP$, выходящий из точки $O$, испытывающий преломление в точке $P$ и пересекающий оптическую ось в точке $O'$. Проведём из центра сферы радиус $CP$. Он ортогонален поверхности сферы $S$, так что для углов $\alpha$ и $\beta$ можно записать закон Снеллиуса:

\begin{equation}
n \sin{\alpha} = n' \sin{\beta}.
\label{eq:snellius_sphere}
\end{equation} 

Обозначим путь луча $OP$ как $u$, а дальнейший путь $PO'$ как $u'$. Найдём их связь из геометрических соображений:

\begin{equation*}
S_{OPC} + S_{PCO'} = S_{O'PO},
\end{equation*}

\begin{align*}
S_{OPC} = -\frac{1}{2}& uR \sin{\alpha}, \\ 
S_{PCO'} = \frac{1}{2}&u'R\sin{\beta}, \\
S_{O'PO} = -\frac{1}{2}&uu'\sin{(\pi - \alpha + \beta).}
\end{align*} 

Из этих равенств находим

\begin{equation*}
-uR\sin{\alpha} + u'R\sin{\beta} = -uu'\sin{(\alpha - \beta).}
\end{equation*}

Подставляем (\ref{eq:snellius_sphere}) и находим

\begin{equation*}
-u + u'\frac{n}{n'} = -\frac{uu'}{R}\Big(\cos{\beta} - \cos{\alpha} \frac{n}{n'}\Big),
\end{equation*}

или

\begin{equation}
\frac{n}{u} - \frac{n'}{u'} = \frac{n\cos{\alpha} - n'\cos{\beta}}{R}.
\label{eq:spherical_surface}
\end{equation}

Это точное соотношение. В случае параксиальных пучков, для которых $|\alpha| \ll 1$, $|\beta| \ll 1$, формула (\ref{eq:spherical_surface}) принимает вид

\begin{equation}
\frac{n}{x} - \frac{n'}{x'} = \frac{n - n'}{R},
\label{eq:spherical_surface1}
\end{equation}

где $x$ — расстояние от точки $O$ до поверхности сферы, $x'$ — расстояние от точки $O'$ до поверхности сферы. Знаки величин $x$ и $x'$ определяются так же, как и для $u$ и $u'$, из того условия, что отсчёт расстояния ведётся от преломляющей поверхности \textit{по направлению лучей}.

Подставив $x = -\infty$ и $x' = \infty$, находим положение заднего и переднего фокусов соответственно:

\begin{align*}
 f' = x' = &\frac{R}{1 - n/n'}, \\ 
 f = x = -R&\frac{n/n'}{1 - n/n'}. \\
\end{align*} 

\subsection{Тонкие линзы}

\textbf{Линза} — это прозрачное тело, изготовленное из прозрачного оптически однородного материала, ограниченное двумя полированными выпуклыми или вогнутыми поверхностями. 

Точки пересечения поверхностей линзы с оптической осью называются \textit{вершинами линзы}. Расстояние $d$ между вершинами линзы называется \textit{толщиной линзы}. Линза считается \textbf{тонкой}, если её толищина мала по сравнению с радиусами кривизны поверхностей: $d \ll R_{1}$, $d \ll R_{2}$. Главные плоскости тонкой линзы совпадают.

Представим линзу как совокупность двух последовательных преломляющих поверхностей. Согласно (\ref{eq:spherical_surface1}), для первой поверхности имеем:

\begin{equation*}
\frac{n_{e}}{x} - \frac{n_{i}}{x_{1}} = \frac{n_{e} - n_{i}}{R_{1}}.
\end{equation*}

Поскольку линза предполагается тонкой, то координата $x_{1}$ промежуточного изображения одинакова относительно обеих поверхностей. Следовательно, для второй поверхности можем записать

\begin{equation*}
\frac{n_{i}}{x_{1}} - \frac{n_{e}}{x'} = \frac{n_{i} - n_{e}}{R_{2}}.
\end{equation*}

Складывая эти равенства, получаем 

\begin{equation}
\frac{1}{x} - \frac{1}{x'} = -(n - 1) \Big(\frac{1}{R_{1}} - \frac{1}{R_{2}}\Big),
\label{eq:lens_formula}
\end{equation}

Здесь мы перешли к относительному показателю преломления $n = n_{i}/n_{e}$.

Фокусные расстояния тонкой линзы:

\begin{align*}
\frac{1}{f'} = (n - &1) \Big(\frac{1}{R_{1}} - \frac{1}{R_{2}}\Big), \\ 
f &= -f'. \\
\end{align*} 

Для двояковыпуклой (или \textit{собирающей}) линзы $R_{1}>0$, $R_{2}<0$ и фокусное расстояние $f' > 0$, $f < 0$. Для двояковогнутой (или \textit{рассеивающей}) линзы всё ровным счётом наоборот.

Заметим, что соотношение (\ref{eq:lens_formula}) можно переписать как 

\begin{equation}
\frac{1}{x} - \frac{1}{x'} = \frac{1}{f}.
\label{eq:another_lens_formula}
\end{equation}

Полученная формула называется \textbf{формулой тонкой линзы}.

\textbf{Оптическую силу линзы} определяют как величину, обратно пропорциональную фокусному расстоянию:

\begin{equation}
D = \frac{1}{f'}.
\label{eq:optical_power}
\end{equation}

Для собирающих линз оптическая сила положительна ($f' > 0$), а для рассеивающих — отрицательна ($f' < 0$). Единица измерения оптической силы — \textit{диоптрия} (оптическая сила такой системы, фокусное расстояние которой $|f'|$ равно одному метру).

Оптическая сила аддитивна (в случае тонких линз).

