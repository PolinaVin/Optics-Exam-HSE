\section{Роль дифракции в приборах, формирующих изображение. Критерий Рэлея (применительно к формированию изображений). Дифракционный предел разрешения телескопа и микроскопа.}

Из-за наличия во Вселенной дифракции Фраунгофера на круглом отверстии, вместо точки при фокусировке света мы получаем небольшое пятнышко, которое называется \textbf{пятном (диском) Эйри}. Его радиус можно рассчитать по следующей формуле:

\begin{equation*}
	\rho_{a} = 1.22 \frac{\lambda}{D} F	
\end{equation*}

Здесь $\lambda$ --- длина волны наблюдаемого излучения, $D$ --- диаметр линзы, $F$ --- ее фокусное расстояние.

Пусть мы наблюдаем некоторый объект, который находится на главной оптической оси. Тогда и его изображение будет находиться там же. Пусть теперь у нас добавляется еще один объект, который находится от первоначального на некотором (небольшом) угловом расстоянии $\psi$. Пятно Эйри от этого объекта в таком случае будет находиться на таком же угловом расстоянии от исходного пятна, а линейное расстояние тогда будет равно $l = F \psi$. 

Для того, чтобы понять, возможно ли эти два объекта разрешить, был введен так называемый \textbf{критерий Рэлея}, который говорит, что минимальное расстояние между объектами, на котором их можно разрешить, оказывается таким, что первый минимум пятна Эйри от одного объекта приходится на нулевой максимум пятна Эйри другого (см рисунок) % Вставить рисунок.

С учетом имеющейся у нас формулы для радиуса пятна Эйри мы получим:

\begin{equation*}
	\psi_{min} F = 1.22 \frac{\lambda}{D} F \qrq \boxed{\psi_{min} = 1.22 \frac{\lambda}{D}}
\end{equation*}

Рассмотрим теперь микроскоп. Предположим, что предмет у нас лежит в какой-то среде с показателем преломления $n$ (так делается, например, в диффузионных микроскопах); по другую сторону линзы у нас, соответственно, воздух, с показателем преломления $n_{air} = 1$. Как уже было сказано ранее, от каждой точки предмета будет получаться не точечное изображение, а пятно Эйри. Введем углы $u_1$ и $u_2$ (см. рисунок). Угол $u_1$ называется апертурным углом объектива. % Вставить рисунок.

Как видно из рисунка, верно следующее:

\begin{equation*}
	\frac{D}{L} \approx 2 u_2 \approx 2 \sin u_2
\end{equation*}

Согласно Аббе, идеальный с точки зрения геометрии микроскоп подчиняется следующему условию (\textbf{условие синусов Аббе}):

\begin{equation*}
	l_1 n_1 \sin u_1 = l_2 n_2 \sin u_2 \qrq l_1 n \sin u_1 = l_2 \frac{D}{2L}
\end{equation*}

Здесь $l_1$ и $l_2$ есть линейные размеры предмета и изображения соответственно.

Ну и с учетом критерия Рэлея, который в данном случае удобно сформулировать как "линейный размер изображения должен получаться не меньше радиуса пятна Эйри", мы получим:

\begin{equation*}
	l_{2 min} = \rho_a = l_{1 min} \frac{2 L n\sin u_1}{D} = 1.22 \frac{\lambda}{D} L \qrq l_{1 min} = 0.61 \frac{\lambda}{n\sin u_1}
\end{equation*}

Выражение $n \sin u_1$ в знаменателе называют \textbf{числовой апертурой микроскопа}.